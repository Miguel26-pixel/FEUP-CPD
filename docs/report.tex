\documentclass{report}

\usepackage{graphicx}
\usepackage{algorithm}
\usepackage{algpseudocode}
\usepackage{float}
\usepackage{adjustbox}
\graphicspath{{graphics/}}

\begin{document}
	\begin{titlepage}
		\centering
		\includegraphics[scale=0.35]{logo_feup.png}\linebreak
		
		\vspace{1cm}
		
		{\scshape \large Bachelor in Informatics and Computing Engineering
		Information}
		
		\vspace {1cm}
		
		{\scshape\Huge Distributed and Partitioned Key-Value Store \par}
		
		\vfill
		
		{\scshape \large Parallel and Distributed Computing}
		
		\vfill
		
		\Large David \textsc{Preda} - up201904726 \\ Fernando
		\textsc{Rego} - up201905951 \\ Miguel \textsc{Amorim} - up201907756
		
		\vspace{1cm}
		
		\today
		
	\end{titlepage}

	\tableofcontents
	
	\chapter{Problem Description}
			\paragraph{} A key-value store is a simple storage system that stores
			arbitrary data objects, the values, each of which is accessed by means
			of a key, very much like in a hash table. To ensure persistency, the data
			items and their keys must be stored in persistent storage, e.g. a hard
			disk drive (HDD) or a solid state disk (SSD), rather than in RAM.
			
			\paragraph{}By distributed, we mean that the data items in the key-value
			store are partitioned among different cluster nodes.
			
			\paragraph{} Our design is loosely based on Amazon's Dynamo, in that it
			uses consistent-hashing to partition the key-value pairs among the 
			different nodes. This will be described later, but we recommend that 
			you read the paper, as it may give you ideas to solve some of the 
			challenges you will find.
			
			\paragraph{}The service is expected be able to handle concurrent requests 
			and to tolerate:
			        1 - node crashes,
			        2 - message loss
	
	\chapter{Membership Service}
	
	        \paragraph{}
	        
	        \paragraph{}
	
	\chapter{Key-value Store}
	
	        \paragraph{}
	        
	        \paragraph{}
	
	\chapter{Replication}
	
	        \paragraph{}
	        
	        \paragraph{}

	\chapter{Fault-Tolerance}
	
	        \paragraph{}
	        
	        \paragraph{}

	\chapter{Thread-pools}
	
	        \paragraph{}
	        
	        \paragraph{}

	\chapter{RMI}
	
	        \paragraph{}
	        
	        \paragraph{}

	\chapter{Test Client}
	
	        \paragraph{}
	        
	        \paragraph{}

	\chapter{Conclusions}
	
	        \paragraph{}This project allowed us to deeply understand some computer concepts related to store .
	        
	        \paragraph{}
			

\end{document}
